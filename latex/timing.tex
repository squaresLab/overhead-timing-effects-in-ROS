\documentclass[conference]{IEEEtran}
\IEEEoverridecommandlockouts
% The preceding line is only needed to identify funding in the first footnote. If that is unneeded, please comment it out.
\usepackage{cite}
\usepackage{amsmath,amssymb,amsfonts}
\usepackage{algorithmic}
\usepackage{graphicx}
\usepackage{textcomp}
\usepackage{xcolor}
\def\BibTeX{{\rm B\kern-.05em{\sc i\kern-.025em b}\kern-.08em
    T\kern-.1667em\lower.7ex\hbox{E}\kern-.125emX}}
    
\newcommand{\todo}[1]{\textcolor{violet}{{\bfseries [[TODO: #1]]}}}
    
\begin{document}

\title{The Effects of Timing Delays on ROS Systems\\
\thanks{Identify applicable funding agency here. If none, delete this.}
}

\author{\IEEEauthorblockN{Deborah S. Katz}
\IEEEauthorblockA{\textit{Computer Science Department} \\
\textit{Carnegie Mellon University}\\
Pittsburgh, PA, USA\\
dskatz@cs.cmu.edu}
\and
\IEEEauthorblockN{2\textsuperscript{nd} Given Name Surname}
\IEEEauthorblockA{\textit{dept. name of organization (of Aff.)} \\
\textit{name of organization (of Aff.)}\\
City, Country \\
email address or ORCID}
\and
\IEEEauthorblockN{3\textsuperscript{rd} Given Name Surname}
\IEEEauthorblockA{\textit{dept. name of organization (of Aff.)} \\
\textit{name of organization (of Aff.)}\\
City, Country \\
email address or ORCID}
\and
\IEEEauthorblockN{4\textsuperscript{th} Given Name Surname}
\IEEEauthorblockA{\textit{dept. name of organization (of Aff.)} \\
\textit{name of organization (of Aff.)}\\
City, Country \\
email address or ORCID}
}

\maketitle

\begin{abstract}
Robotics and other cyber-physical systems interact with the real world.
Events in the real world can be unpredictable and sometimes cause delays.
Interacting with the real world sometimes involves the software waiting for
external events.
The architecture of some robotics and cyber-physical systems allows these
systems to absorb many delays without disrupting intended behavior, while
some delays, especially those that occur in critical sections of execution,
are too much for these systems to absorb. 
Furthermore, system monitoring is a set of techniques that can help to understand
when and whether systems are behaving as intended.
These techniques can add unacceptable levels of overhead in many 
circumstances.
However, the overhead of system monitoring techniques may be absorbed into normal execution for some robotics and cyber-physical systems.

This paper analyzes the circumstances under which artificially-inserted 
timing delays have an observable effect on robotics systems.




\end{abstract}

\begin{IEEEkeywords}
robotics, cyber-physical systems, software monitoring, software quality
\end{IEEEkeywords}

\section{Introduction}
Robotics and cyber-physical systems are particularly prone to variability in operating conditions because of their interaction with the real world and the unpredictable conditions therein. Some of these delays show up as timing delays in execution or communication within the system. However, many of these systems have architectures that have the ability to absorb some timing delays, as they are constructed to wait for physical events.

In addition, software monitoring techniques can cause execution delays. These techniques may be useful to evaluate whether software is behaving as intended, but in many legacy pieces of software, they cause unacceptable overhead. I hypothesize that the same properties that allow robotics and cyber-physical systems to absorb timing delays that occur due to real-world unpredictability allow these systems to absorb some of the delays that would be caused by program monitoring.

It is desirable to get a more precise understanding of the amount of delay that these systems can absorb, to determine to what extent they can tolerate software monitoring. 
\todo{I definitely had a better explanation for this reasoning before. Use that instead.}

\section{Background and Motivation}
\label{sec:background}
\todo{Fill in background}

\section{Approach}
\label{sec:approach}
\todo{Fill in approach}


\subsection{Methodology}
\label{sec:methodology}
\todo{As usual, I've conflated approach and methodology. Separate these back out.}

To evaluate the extent to which timing delays deform the observable execution of a robotic system I execute the following experiments:

I run each robot over a series of missions. Each mission is represented as a series of destinations in three dimensional space (two dimensional space for robots that move in only two dimensions), with the final destination being a return to the origin point.

For each combination of robot and mission, I run several \emph{nominal} executions to establish a baseline for what robot behavior should look like in these executions. These are run in simulation on unmodified code.

For the \emph{experimental} executions, I add artificial delays to the execution of the robot code.

For the ArduCopter experiments, the artificial delays are introduced before return statements in the code. For each return statement in a .cpp file in the <directory name> directory, a delay may be placed in the source code before the return statement. The choice of whether to insert a delay was determined probabilistically, with a weighted coin flip. Different modified versions of the code were created, each of which had (a) a fixed coin flip weight and (b) fixed delay amount added at each delay location. The weights for the weighted coin flip ranged from 0.1 to 1.0, and the delays ranged from X to Y.

For the ROS experiments, the artificial delays are introduced at communications barriers on ROS topics. \todo{TODO: insert discussion of how we're introducing delays at ROS topics and what this means in the ROS architecture}
To give a simplified overview of the architecture of ROS-based systems, these systems consist of various nodes which communicate with each other on a bus. A publish-subscribe system determines which nodes receive messages. Each node that publishes messages publishes to a \emph{topic} while each node that receives a message receives by subscribing to a \emph{topic}. We insert delays on these topics by intercepting messages using topic renaming. \todo{the explanation here probably needs to be better}. The topics to delay were chosen by X \todo{principled explanation of how I picked which topics to delay}. Delays ranged from X to Y and were inserted for every message in a topic. \todo{do I need to do this probabilistically as well?}



I evaluate the following metrics.
\begin{itemize}

\item Whether each execution executes each waypoint and returns home
\item Euclidean distance metrics: these metrics are based on the position in 3d space of the deformed execution versus the nominal executions.
\begin{itemize}

\item the Euclidean distance between the final position of the robot in the representative nominal and each deformed execution
\item given aligned time series between the representative nominal execution and each deformed execution, the greatest and the average Euclidean difference between each position on the path
\item based on the closest distance from each waypoint
\begin{itemize}

\item the sum of closest distances from each waypoint
\item the average of the closest distances from each waypoint
\item the greatest closest distance from each waypoint
\end{itemize}


\end{itemize}
\item Timeliness metrics
\begin{itemize}

\item The amount of time before completion (either successfully or unsuccessfully)
\item Total amount of time taken to reach each waypoint (`reach' defined as when the system issues the instruction to go to the next waypoint)
\end{itemize}

\item \todo{discussion of more complex metrics}
\end{itemize}


To determine a representative nominal execution, I take the medoid of the time series representing the 3d positions of the robot in all nominal executions of a particular mission (without artificial delays).

I evaluate the following research questions.
\begin{description}
\item[\textbf{RQ1:}] To what extent do the presence of timing delays in robot systems have an effect on observable behavior as defined by a set of performance metrics?
\item[\textbf{RQ2:}] Are certain kinds of robotics components more robust to resilient to timing delays?
\item[\textbf{RQ3:}] How could software-based simulation platforms be improved
  to encourage developers to use simulation for software-based testing?
\item[\textbf{RQ4:}] Under what circumstances do timing delays lead to system crashes?
\end{description}

\section*{Acknowledgments}

\todo{add acknowledgements}

\section*{References}
\todo{add bibliography}

\end{document}
